\chapter{\ifproject%
\ifenglish Experimentation and Results\else การทดลองและผลลัพธ์\fi
\else%
\ifenglish System Evaluation\else การประเมินระบบ\fi
\fi}

\section{การประเมินผลและการวิเคราะห์แบบจำลอง (Model Evaluation and Analysis)}

\subsection{การตรวจสอบความถูกต้องของแบบจำลอง (Model Validation)}
ตรวจสอบว่าแบบจำลองที่สร้างขึ้นสามารถทำงานได้ตรงตามทฤษฎีหรือหลักการที่นำมาใช้ เช่น \textit{Nash Equilibrium} และ \textit{Prisoner’s Dilemma}  
ทำการแสดงผลการรัน (\textit{simulation results}) ของแต่ละกรณี เช่น \textit{payoff function} และ \textit{tree-based attack-defense model}  
จากนั้นเปรียบเทียบค่าผลลัพธ์ที่ได้กับค่าที่คาดไว้ (\textit{Expected Values}) เพื่อยืนยันความถูกต้องของแบบจำลอง

\subsection{การวิเคราะห์ผล (Result Analysis)}
วิเคราะห์กรณีต่าง ๆ ของผู้โจมตีและผู้ป้องกัน เพื่อประเมินว่าค่าผลตอบแทน (\textit{Payoff}) และความเสียหาย (\textit{Damage}) เป็นไปตามที่คาดไว้หรือไม่  
ใช้ตารางหรือกราฟเพื่อแสดงความแตกต่างของผลลัพธ์ในแต่ละสถานการณ์ (\textit{Scenario})  
พร้อมทั้งแสดงให้เห็นว่าแนวทาง \textit{Greedy Algorithm} หรือ \textit{DFS/BFS} มีประสิทธิภาพเพียงใดในการเลือกเส้นทางหรือกลยุทธ์ที่เหมาะสมที่สุด

\subsection{การเปรียบเทียบกับวิธีอื่น (Comparison)}
เปรียบเทียบแบบจำลองที่สร้างขึ้นกับวิธีการหรือโมเดลอื่นที่คล้ายกัน เพื่อระบุข้อดีและข้อจำกัดของแบบจำลองที่พัฒนา เช่น  
การเปรียบเทียบระหว่าง \textit{Greedy Approach} กับ \textit{Exhaustive Search (DFS แบบเต็ม)} ในด้านเวลาในการคำนวณและความแม่นยำของผลลัพธ์  

\textbf{ข้อจำกัดของโมเดล (Limitations)}  
ระบุข้อจำกัดที่อาจเกิดขึ้น เช่น
\begin{itemize}
  \item จำนวนโหนดของต้นไม้ (\textit{Tree}) ที่มากเกินไปส่งผลให้เวลาในการคำนวณ (\textit{Computation Time}) สูง
  \item การประมาณค่า \textit{Payoff} หรือ \textit{Defense Cost} อาจไม่ครอบคลุมทุกสถานการณ์ในโลกจริง
\end{itemize}

\subsection{ข้อเสนอแนะในการปรับปรุง (Suggestions for Improvement)}
เสนอแนวทางปรับปรุงเพื่อให้แบบจำลองมีความยืดหยุ่นและแม่นยำยิ่งขึ้น เช่น  
การนำแนวคิดเชิง \textit{Heuristic} หรือ \textit{Machine Learning} มาใช้ในการเลือก \textit{Defense Strategy}  
การขยายแบบจำลองให้รองรับผู้โจมตีหลายคน หรือการจำลองหลายรอบเพื่อวิเคราะห์ความเสี่ยงเชิงสถิติ

\subsection{สรุปผลการประเมิน (Evaluation Summary)}
สรุปผลการประเมินว่าแบบจำลองสามารถตอบโจทย์วัตถุประสงค์ของงานวิจัยได้หรือไม่  
ระบุจุดแข็ง เช่น ความสามารถในการคำนวณ \textit{Payoff Function} และการวิเคราะห์ \textit{Defense Strategy} ได้อย่างชัดเจน  
รวมถึงระบุจุดอ่อน เช่น เวลาในการคำนวณ (\textit{Computation Time}) ที่สูง หรือสมมติฐานบางประการที่อาจไม่สอดคล้องกับสถานการณ์จริง
