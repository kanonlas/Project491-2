\maketitle
\makesignature

\ifproject
\begin{abstractTH}

โครงงานนี้จัดทำขึ้นเพื่อนำเสนอการประยุกต์ใช้ทฤษฎีเกมและการคำนวณเชิงควอนตัม โดยนำไปแก้ไขปัญหาด้านความมั่นคงปลอดภัยไซเบอร์ ซึ่งมองจากสองมุมมองหลัก อันได้แก่ ฝ่ายป้องกันและฝ่ายโจมตี ภายใต้บริบทของทฤษฎีเกม โดยสมมติให้ผู้เล่นทั้งสองฝ่ายประกอบไปด้วย ผู้ป้องกัน และ ผู้โจมตี 
โครงงานนี้ได้พัฒนารูปแบบของระบบความมั่นคงปลอดภัยไซเบอร์แบบหลายชั้น ผ่านรูปแบบโครงสร้างต้นไม้ (tree) จากทฤษฎีกราฟ ซึ่งโหนดแต่ละจุดจะแทนรางวัลหรือข้อมูลที่ผู้โจมตีจะได้รับหลังจากเลือกโหนดนั้น เเละผู้ป้องกันต้องปกป้องเเต่ละโหนดโดยการลดจำนวนรางวัลที่ผู้โจมตีจะได้รับ และเส้นเชื่อม (edge) จะแทนต้นทุนที่ใช้ในการโจมตีเพื่อเข้าถึงรางวัลนั้น ๆ หลังจากนั้นจะทำการศึกษาและวิเคราะห์ปัญหาต่าง ๆ บนแบบจำลองดังกล่าว พร้อมทั้งประยุกต์ใช้แนวคิด ดุลยภาพแนช (Nash Equilibrium) และประมวลผลของผลลัพธ์ด้วยกระบวนการ ควอนตัมแอนนีลลิง (Quantum Annealing) เพื่อหาคำตอบที่เหมาะสมที่สุดสำหรับปัญหาในแบบจำลอง
\LaTeX{} เพื่อช่วยให้นักศึกษาเขียนรายงานได้อย่างสะดวกและรวดเร็วมากยิ่งขึ้น
\end{abstractTH}

\begin{abstract}
The abstract would be placed here. It usually does not exceed 350 words
long (not counting the heading), and must not take up more than one (1) page
(even if fewer than 350 words long).

Make sure your abstract sits inside the \texttt{abstract} environment.
\end{abstract}

\iffalse
\begin{dedication}
This document is dedicated to all Chiang Mai University students.

Dedication page is optional.
\end{dedication}
\fi % \iffalse

\begin{acknowledgments}
Your acknowledgments go here. Make sure it sits inside the
\texttt{acknowledgment} environment.

\acksign{2020}{5}{25}
\end{acknowledgments}%
\fi % \ifproject

\contentspage

\ifproject
\figurelistpage

\tablelistpage
\fi % \ifproject

% \abbrlist % this page is optional

% \symlist % this page is optional

% \preface % this section is optional
