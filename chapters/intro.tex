\chapter{\ifenglish Introduction\else บทนำ\fi}

\section{\ifenglish Project rationale\else ที่มาของโครงงาน\fi}
ระบบคอมพิวเตอร์และเครือข่ายอินเทอร์เน็ตที่ใช้งานโดยทั่วไปก็มีบทบาทสำคัญต่อทุกภาคส่วนของสังคม 
แต่ในขณะเดียวกัน ความมั่นคงปลอดภัยทางไซเบอร์ (Cybersecurity) ก็กลายเป็นความท้าทายประการหนึ่ง 
ซึ่งมีความซับซ้อนและยากต่อการจัดการ อันเนื่องจากผู้โจมตีมักจะพัฒนากลยุทธ์ใหม่ ๆ อยู่อย่างเสมอ 
เพื่อนำมาแทรกแซงระบบซอฟต์แวร์ที่ถูกพัฒนาขึ้นมาใหม่ตามกาลเวลา 
ทำให้ในขณะเดียวกันนั่นเอง ผู้ป้องกันก็จำเป็นที่จะต้องหาวิธีการที่มีประสิทธิภาพในการรับมือ โครงงานวิจัยนี้จึงนำเสนอการใช้ \textbf{ทฤษฎีเกม (Game Theory)} 
มาเป็นกรอบแนวคิดในการแก้ปัญหา โดยมองว่าการโจมตีและการป้องกันเป็นเกมที่มีผู้เล่นสองฝ่าย ได้แก่ 
\textbf{ผู้โจมตี (Attacker)} และ \textbf{ผู้ป้องกัน (Defender)}  

นอกจากนี้ โครงงานยังมีการประยุกต์ใช้ \textbf{การคำนวณเชิงควอนตัม (Quantum Computing)} 
โดยเฉพาะการแก้ปัญหาแบบ \textbf{ควอนตัมแอนนีลลิง (Quantum Annealing)} 
เพื่อนำมาหาคำตอบที่เหมาะสมที่สุดในสถานการณ์ที่ซับซ้อน 
ซึ่งวิธีการนี้จะสามารถช่วยลดเวลาในการคำนวณและช่วยหาผลลัพธ์ที่มีประสิทธิภาพ 
กว่าการคำนวณแบบดั้งเดิม  
\section{\ifenglish Objectives\else วัตถุประสงค์ของโครงงาน\fi}
\begin{enumerate}
        \item เพื่อสร้างความเข้าใจในการประยุกต์การใช้ทฤษฎีเกมกับปัญหาด้านความมั่นคงปลอดภัยไซเบอร์ ผ่านการออกแบบให้อยู่ในรูปสมการเชิงควอนตัม
        \item เพื่อออกแบบแบบจำลองระบบความมั่นคงปลอดภัยไซเบอร์แบบหลายชั้น โดยใช้โครงสร้างต้นไม้ (Tree) จากทฤษฎีกราฟ
        \item เพื่อประยุกต์ใช้แนวคิดดุลยภาพแนช (Nash Equilibrium) ในการวิเคราะห์ผลลัพธ์ของเกมระหว่างผู้โจมตีและผู้ป้องกันผ่านมูลค่าของผลลัพธ์ที่แต่ละฝ่ายได้รับ
        \item เพื่อประยุกต์ใช้การคำนวณเชิงควอนตัมแบบ ควอนตัมแอนนีลลิง (Quantum Annealing) ในการแก้ปัญหาบนแบบจำลองโครงสร้างต้นไม้ที่สร้างขึ้น
\end{enumerate}

\section{\ifenglish Project scope\else ขอบเขตของโครงงาน\fi}

\subsection{\ifenglish Hardware scope\else ขอบเขตด้านฮาร์ดแวร์\fi}
ใช้เพียงคอมพิวเตอร์หรือโน้ตบุ๊กทั่วไปที่สามารถเชื่อมต่ออินเทอร์เน็ตได้ โดยไม่ลงลึกถึงการพัฒนาหรือใช้งานฮาร์ดแวร์ควอนตัมโดยตรงอย่างควอนตัมโพรเซสเซอร์ (Quantum Processor)

\subsection{\ifenglish Software scope\else ขอบเขตด้านซอฟต์แวร์\fi}
เนื่องจากประเทศไทย ไม่มีเครื่องควอนตัมคอมพิวเตอร์ที่สามารถใช้งานเพื่อประมวลผลได้ ดังนั้นจึงจำเป็นที่จะต้องมีการประมวลผลผ่านเครื่องคอมพิวเตอร์แบบคลาสสิคแพลตฟอร์ม Quantum Simulation ที่มีให้บริการออนไลน์ เช่น D-Wave Leap และใช้ไลบรารีโอเพนซอร์สที่เกี่ยวข้อง เช่น Dimod สำหรับการคำนวณเชิงควอนตัม โดยภาษาโปรแกรมที่ใช้เป็นหลักคือ Python สำหรับการทดลองนี้

\subsection{\ifenglish Theory and software scope\else  ขอบเขตด้านทฤษฎีและการจำลอง\fi}
มุ่งเน้นการสร้างแบบจำลองระบบความมั่นคงปลอดภัยไซเบอร์เชิงลำดับชั้นด้วยโครงสร้างต้นไม้ (Tree Structure) โดยโหนด (Node) แทนทรัพยากรหรือรางวัลของระบบ ที่ฝ่ายโจมตีต้องการ โดยอาจจะเป็นข้อมูลหรือทรัพยากรที่สำคัญในบริบทของไซเบอร์ และเส้นเชื่อม (Edge) แทนต้นทุนของการโจมตีเพื่อเข้าถึงรางวัลนั้น ๆ ซึ่งจะนำมาวิเคราะห์ปัญหาผ่านแนวคิด ดุลยภาพแนช (Nash Equilibrium) และ ควอนตัมแอนนีลลิง (Quantum Annealing) โดยจะเน้นไปในบริบทของ ระบบความปลอดภัยเครือข่าย (Network Security) และระบบที่เกี่ยวข้องอื่น ๆ กับความปลอดภัยไซเบอร์

\section{\ifenglish Expected outcomes\else ประโยชน์ที่ได้รับ\fi}
\begin{enumerate}
    \item ได้ความรู้และความเข้าใจในการประยุกต์ใช้ทฤษฎีเกมกับปัญหาด้านความมั่นคงปลอดภัยไซเบอร์
    \item  เข้าใจหลักการและศักยภาพของการคำนวณเชิงควอนตัม โดยเฉพาะการแก้ปัญหาเพื่อหากระบวนการหาคำตอบที่ดีที่สุด (Optimization)
    \item สามารถนำแบบจำลองที่พัฒนาขึ้นมาใช้เป็นแนวทางในการศึกษาเชิงลึกด้านการป้องกันและการโจมตีภายในระบบความมั่นคงปลอดภัยไซเบอร์
    \item เป็นพื้นฐานให้กับงานวิจัยในอนาคตที่เกี่ยวข้องกับการผสมผสานกันระหว่างทฤษฎีเกมและการคำนวณเชิงควอนตัม
\end{enumerate}

\section{\ifenglish Technology and tools\else เทคโนโลยีและเครื่องมือที่ใช้\fi}

\subsection{\ifenglish Hardware technology\else เทคโนโลยีด้านฮาร์ดแวร์\fi}
คอมพิวเตอร์หรือโน้ตบุ๊กส่วนตัวที่สามารถเชื่อมต่ออินเทอร์เน็ตได้
\subsection{\ifenglish Software technology\else เทคโนโลยีด้านซอฟต์แวร์\fi}
\begin{enumerate}
\item Web Browser สำหรับเข้าใช้งาน Quantum Platform เช่น D-Wave

\item ไลบรารี Python สำหรับ Quantum Simulation เช่น dimod
\end{enumerate}


\newif\ifBudget
\Budgetfalse % หรือ \Budgettrue ถ้าอยากให้แสดงภาษาอังกฤษ

\section{\ifBudget Allocation Plan\else แผนการใช้งบประมาณ\fi}
โครงงานนี้ ไม่จำเป็นต้องใช้งบประมาณ อันเนื่องมาจากอาศัยการใช้โปรแกรม, ไลบรารี, และแพลตฟอร์มจำลองการคำนวณเชิงควอนตัมที่เปิดให้ใช้ฟรีบนอินเทอร์เน็ต เช่น D-Wave รวมถึงเครื่องมือที่สามารถติดตั้งและใช้งานได้ในคอมพิวเตอร์ส่วนตัวโดยทั่วไปได้โดยไม่เสียค่าใช้จ่าย

\section{\ifenglish Project plan\else แผนการดำเนินงาน\fi}

\begin{plan}{6}{2025}{3}{2026}
    \planitem{6}{2025}{6}{2025}{รวบรวมสมาชิกและกำหนดหัวเรื่องโครงงาน}
    \planitem{6}{2025}{7}{2025}{ศึกษาทฤษฎีและงานวิจัยที่เกี่ยวข้อง}
    \planitem{7}{2025}{8}{2025}{พัฒนาแบบจำลองขั้นต้นของอัลกอริทึมโดยใช้ปัญหาของทฤษฎีเกมขั้นพื้นฐาน}
    \planitem{9}{2025}{10}{2025}{รวบรวมข้อมูลและออกแบบแบบจำลองในบริบทปัญหาของไซเบอร์}
    \planitem{11}{2025}{1}{2026}{นำอัลกอริทึมไปรันบนแพลตฟอร์มจำลองเพื่อทดสอบหาค่าผลลัพธ์}
    \planitem{1}{2026}{2}{2026}{บันทึกผล วิเคราะห์ผลดีผลเสีย และสรุปผล}
    \planitem{2}{2026}{3}{2026}{จัดทำรายงาน โปสเตอร์ และสื่อนำเสนอ}
\end{plan}


\section{\ifenglish Roles and responsibilities\else บทบาทและความรับผิดชอบ\fi}

เนื่องจากหัวข้อโครงงานต้องอาศัยศาสตร์แขนงของความรู้ที่หลากหลาย ทั้งในด้านของ ความมั่นคงปลอดภัยทางไซเบอร์, ทฤษฎีเกม และการคำนวณเชิงควอนตัม ทำให้สมาชิกในกลุ่มทุกคนต่างก็มีส่วนร่วมในทุกขั้นตอน โดยจะเน้นการทำงานร่วมกันในแต่ละขั้นตอนของโครงงานพร้อม ๆ กัน ช่วยเหลือกันแบ่งหน้าที่ในการค้นคว้า ออกแบบ และตรวจสอบผลลัพธ์ เพื่อให้โครงงานดำเนินไปอย่างถูกต้องและมีประสิทธิภาพมากที่สุด ซึ่งสามารถแบ่งเป็นหลัก ๆ ได้ดังนี้
\begin{enumerate}
\item นางสาวกนลลัส รัตนภาค รับผิดชอบหน้าที่หลักในการรวบรวมงานวิจัยที่เกี่ยวข้อง และสรุปองค์ความรู้ที่จำเป็นต้องใช้ในโครงงาน ทั้งทฤษฎีเกม สมการเชิงควอนตัม รวมไปถึง ออกแบบโครงสร้างแผนภาพต้นไม้ทั้งหมดก่อนนำไปแปลงเป็นสมการทางคณิตศาสตร์

\item นายธีระพันธุ์ พันธุ์วรรธนะสิน ทำหน้าที่ในการแปลงปัญหาในทฤษฎีเกม ทั้งในแบบจำลองตั้งต้นกับปัญหาทางไซเบอร์ซึ่งเป็นแผนภาพต้นไม้ ให้ออกมาอยู่ในรูปแบบของสมการคณิตศาสตร์ที่มีตัวแปรและอัลกอริทึมเพื่อที่จะสามารถแปลงเป็นโค้ด Python ได้
\item นางสาวแก้วตา ลุงต๊ะ รับผิดชอบหน้าที่หลักในการเขียนโค้ด Python ของอัลกอริทึม โดยอาศัยจากสมการคณิตศาสตร์ที่ได้แปลงจากแผนภาพต้นไม้มาแล้ว นำไปเขียนรูปแบบใหม่ในรูปของโค้ด Python โดยอาศัยไลบลารี dimod เพื่อวัดดูประสิทธิภาพและคำตอบของอัลกอริทึมในตอนสุดท้าย
\end{enumerate}

\section{\ifenglish%
Impacts of this project on society, health, safety, legal, and cultural issues
\else%
ผลกระทบด้านสังคม สุขภาพ ความปลอดภัย กฎหมาย และวัฒนธรรม
\fi}

การประยุกต์ใช้ทฤษฎีเกมเข้ากับความมั่นคงปลอดภัยไซเบอร์ เป็นการส่งเสริมการพัฒนาองค์ความรู้ใหม่ ๆ ที่สามารถต่อยอดไปสู่การป้องกันการโจมตีทางไซเบอร์ในอนาคต ซึ่งนำมาสนับสนุนการพัฒนาเทคนิคการวิเคราะห์ความเสี่ยงที่จะถูกโจมตีเพื่อช่วยเพิ่มความปลอดภัยของระบบเครือข่าย และอีกทั้งยังช่วยให้หน่วยงานหรือองค์กรที่เป็นเจ้าของระบบความปลอดภัยสามารถนำแนวคิดนี้ไปปรับใช้เพื่อป้องกันความเสียหายจากการโจมตี ผ่านการออกแบบเชิงนโยบายและกฎหมายให้รัดกุมมากยิ่งขึ้น

