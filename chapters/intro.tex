\chapter{\ifenglish Introduction\else บทนำ\fi}

\section{\ifenglish Project rationale\else ที่มาของโครงงาน\fi}
ระบบคอมพิวเตอร์และเครือข่ายอินเทอร์เน็ตที่ใช้งานโดยทั่วไปก็มีบทบาทสำคัญต่อทุกภาคส่วนของสังคม 
แต่ในขณะเดียวกัน ความมั่นคงปลอดภัยทางไซเบอร์ (Cybersecurity) ก็กลายเป็นความท้าทายประการหนึ่ง 
ซึ่งมีความซับซ้อนและยากต่อการจัดการ อันเนื่องจากผู้โจมตีมักจะพัฒนากลยุทธ์ใหม่ ๆ อยู่อย่างเสมอ 
เพื่อนำมาแทรกแซงระบบซอฟต์แวร์ที่ถูกพัฒนาขึ้นมาใหม่ตามกาลเวลา 
ทำให้ในขณะเดียวกันนั่นเอง ผู้ป้องกันก็จำเป็นที่จะต้องหาวิธีการที่มีประสิทธิภาพในการรับมือ  

โครงงานวิจัยนี้จึงนำเสนอการใช้ \textbf{ทฤษฎีเกม (Game Theory)} 
มาเป็นกรอบแนวคิดในการแก้ปัญหา โดยมองว่าการโจมตีและการป้องกันเป็นเกมที่มีผู้เล่นสองฝ่าย ได้แก่ 
\textbf{ผู้โจมตี (Attacker)} และ \textbf{ผู้ป้องกัน (Defender)}  

นอกจากนี้ โครงงานยังมีการประยุกต์ใช้ \textbf{การคำนวณเชิงควอนตัม (Quantum Computing)} 
โดยเฉพาะการแก้ปัญหาแบบ \textbf{ควอนตัมแอนนีลลิง (Quantum Annealing)} 
เพื่อนำมาหาคำตอบที่เหมาะสมที่สุดในสถานการณ์ที่ซับซ้อน 
ซึ่งวิธีการนี้จะสามารถช่วยลดเวลาในการคำนวณและช่วยหาผลลัพธ์ที่มีประสิทธิภาพ 
กว่าการคำนวณแบบดั้งเดิม  
\section{\ifenglish Objectives\else วัตถุประสงค์ของโครงงาน\fi}
\begin{enumerate}
        \item เพื่อสร้างความเข้าใจในการประยุกต์การใช้ทฤษฎีเกมกับปัญหาด้านความมั่นคงปลอดภัยไซเบอร์ ผ่านการออกแบบให้อยู่ในรูปสมการเชิงควอนตัม
        \item เพื่อออกแบบแบบจำลองระบบความมั่นคงปลอดภัยไซเบอร์แบบหลายชั้น โดยใช้โครงสร้างต้นไม้ (Tree) จากทฤษฎีกราฟ
        \item เพื่อประยุกต์ใช้แนวคิดดุลยภาพแนช (Nash Equilibrium) ในการวิเคราะห์ผลลัพธ์ของเกมระหว่างผู้โจมตีและผู้ป้องกันผ่านมูลค่าของผลลัพธ์ที่แต่ละฝ่ายได้รับ
        \item เพื่อประยุกต์ใช้การคำนวณเชิงควอนตัมแบบ ควอนตัมแอนนีลลิง (Quantum Annealing) ในการแก้ปัญหาบนแบบจำลองโครงสร้างต้นไม้ที่สร้างขึ้น
\end{enumerate}

\section{\ifenglish Project scope\else ขอบเขตของโครงงาน\fi}

\subsection{\ifenglish Hardware scope\else ขอบเขตด้านฮาร์ดแวร์\fi}

\subsection{\ifenglish Software scope\else ขอบเขตด้านซอฟต์แวร์\fi}

\section{\ifenglish Expected outcomes\else ประโยชน์ที่ได้รับ\fi}

\section{\ifenglish Technology and tools\else เทคโนโลยีและเครื่องมือที่ใช้\fi}

\subsection{\ifenglish Hardware technology\else เทคโนโลยีด้านฮาร์ดแวร์\fi}

\subsection{\ifenglish Software technology\else เทคโนโลยีด้านซอฟต์แวร์\fi}

\section{\ifenglish Project plan\else แผนการดำเนินงาน\fi}

\begin{plan}{6}{2020}{2}{2021}
    \planitem{7}{2020}{8}{2020}{ศึกษาค้นคว้า}
    \planitem{8}{2020}{1}{2021}{ชิล}
    \planitem{2}{2021}{2}{2021}{เผา}
    \planitem{12}{2019}{1}{2022}{ทดสอบ}
\end{plan}

\section{\ifenglish Roles and responsibilities\else บทบาทและความรับผิดชอบ\fi}
อธิบายว่าในการทำงาน นศ. มีการกำหนดบทบาทและแบ่งหน้าที่งานอย่างไรในการทำงาน จำเป็นต้องใช้ความรู้ใดในการทำงานบ้าง

\section{\ifenglish%
Impacts of this project on society, health, safety, legal, and cultural issues
\else%
ผลกระทบด้านสังคม สุขภาพ ความปลอดภัย กฎหมาย และวัฒนธรรม
\fi}

แนวทางและโยชน์ในการประยุกต์ใช้งานโครงงานกับงานในด้านอื่นๆ รวมถึงผลกระทบในด้านสังคมและสิ่งแวดล้อมจากการใช้ความรู้ทางวิศวกรรมที่ได้
