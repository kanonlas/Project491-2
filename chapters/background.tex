\chapter{\ifenglish Background Knowledge and Theory\else ทฤษฎีที่เกี่ยวข้อง\fi}

\section{อัลกอริทึม (Algorithm)}
อัลกอริทึม (Algorithm) คือชุดขั้นตอนหรือลำดับคำสั่งที่ใช้แก้ปัญหาหรือดำเนินการในคอมพิวเตอร์หรือระบบต่างๆ โดยมีขั้นตอนที่ชัดเจนและเป็นระบบ เพื่อให้ได้ผลลัพธ์ที่ถูกต้องตามที่ต้องการ อัลกอริทึมคือกระบวนการแก้ปัญหาที่สามารถอธิบายเป็นขั้นตอนอย่างละเอียด เมื่อได้รับข้อมูลนำเข้า จะต้องให้ผลลัพธ์ที่ถูกต้องและมีประสิทธิภาพ อัลกอริทึมที่ดีจะต้องมีความชัดเจนไม่คลุมเครือ ซึ่งการแก้ปัญหาโดยใช้อัลกอริทึมตรงข้ามกับการแก้ปัญหาโดยใช้สามัญสํานึก

\section{Cybersecurity Model}
Cybersecurity Model คือแบบจำลองเพื่อวิเคราะห์และจำลองปฏิสัมพันธ์ระหว่างผู้โจมตี (attacker) และผู้ป้องกัน (defender) ในโลกไซเบอร์ โดยทั่วไปจะมีการใช้เกมสองผู้เล่นที่เป็น zero-sum game ซึ่งผู้โจมตีพยายามหาช่องโหว่เพื่อโจมตี ส่วนผู้ป้องกันพยายามลดความเสียหายและป้องกันระบบ โดยผู้เล่นทั้งสองฝ่ายสามารถเลือกกลยุทธ์หลายระดับ เช่น ระดับไม่โจมตีหรือไม่ป้องกัน, ระดับโจมตี/ป้องกันต่ำ, และระดับโจมตี/ป้องกันสูง

\section{Graph Theory}
ทฤษฎีกราฟ (Graph Theory) คือสาขาหนึ่งของคณิตศาสตร์และวิทยาการคอมพิวเตอร์ที่ศึกษาถึงคุณสมบัติและการใช้งานของกราฟ ซึ่งเป็นโครงสร้างข้อมูลที่ประกอบด้วยจุดยอด (Vertices) และเส้นเชื่อม (Edges) ที่เชื่อมต่อระหว่างจุดยอดเหล่านั้น กราฟเป็นแบบจำลองทางคณิตศาสตร์ที่ใช้แทนความสัมพันธ์หรือโครงสร้างของเครือข่ายต่างๆ

\subsection{Tree Structure}
โครงสร้างต้นไม้ (Tree structure) ในทฤษฎีกราฟ หมายถึงกราฟที่ไม่มีวงจร (acyclic) และเป็นกราฟที่เชื่อมต่อกันทั้งหมด (connected) โดยลักษณะสำคัญของต้นไม้คือ ในกราฟต้นไม้จะมีเส้นเชื่อม (edges) เท่ากับจำนวนจุดเชื่อม (vertices) ลบหนึ่ง 
\[
|E| = |V| - 1
\]
มีเส้นทางเชื่อมโยงเดียวและไม่ซ้ำกันระหว่างจุดเชื่อมใดๆ สองจุด ต้นไม้ไม่มีวงจรใดๆ หากตัดเส้นเชื่อมใดเส้นหนึ่งออก จะทำให้กราฟไม่เชื่อมต่อ (แตกออกเป็นส่วนย่อย)  

ต้นไม้ถือเป็นกราฟสองกลุ่ม (bipartite graph) และกราฟแผนที่ (planar graph) จุดที่มีเส้นเชื่อมแค่หนึ่งเส้น เรียกว่าใบไม้ (leaf หรือ terminal vertex) มีจุดศูนย์กลาง (center) หรือจุดสมดุล (centroid) ที่แบ่งต้นไม้ได้อย่างสมดุล  

โดยทั่วไป นิยามของต้นไม้คือกราฟที่เชื่อมต่อกันและไม่มีวงจร ซึ่งถือเป็นโครงสร้างพื้นฐานในหลายด้าน เช่น โครงสร้างข้อมูล คอมพิวเตอร์ เครือข่าย และระบบไฟฟ้า เป็นต้น

\section{Game Theory}
Game Theory คือศาสตร์ที่ศึกษาแบบจำลองทางคณิตศาสตร์ของสถานการณ์ที่มีการโต้ตอบเชิงกลยุทธ์ระหว่างผู้เล่นหลายฝ่าย ซึ่งมักใช้วิเคราะห์การตัดสินใจของผู้เล่นที่มีเหตุผลในสถานการณ์แข่งขันหรือร่วมมือกัน โดยตั้งสมมติฐานว่าผู้เล่นแต่ละคนจะพยายามเพิ่มผลประโยชน์ของตนเองจากกลยุทธ์ที่เลือกใช้ โดยในหนึ่งเกม จะประกอบไปด้วย

\begin{enumerate}
  \item ผู้เล่น (Players) คือผู้มีส่วนร่วมในการตัดสินใจ
  \item กลยุทธ์ (Strategies) คือทางเลือกที่ผู้เล่นสามารถเลือกใช้ได้
  \item ผลตอบแทน (Payoffs) คือผลลัพธ์หรือรางวัลที่ผู้เล่นได้รับจากการเลือกกลยุทธ์
  \item กติกาของเกม (Rules of the game) คือเงื่อนไขที่กำหนดว่าผู้เล่นโต้ตอบกันอย่างไร
\end{enumerate}

\subsection{เกมเชิงกลยุทธ์ (Strategic Game)}
เกมเชิงกลยุทธ์ (Strategic Game) สามารถนิยามได้เป็นพจน์
\[
G = (N, S, u)
\]
โดยที่
\begin{enumerate}
  \item \( N = \{1, 2, \ldots, n\} \) คือเซตของผู้เล่น
  \item \( S = S_1 \times S_2 \times \cdots \times S_n \) คือเซตของกลยุทธ์
  \item \( u = (u_1, u_2, \ldots, u_n) \) คือฟังก์ชันผลตอบแทนของผู้เล่นแต่ละคน
\end{enumerate}

\subsection{Nash Equilibrium}
Nash Equilibrium คือสภาวะที่ไม่มีผู้เล่นคนใดสามารถได้ผลประโยชน์มากขึ้นโดยการเปลี่ยนกลยุทธ์ของตนเอง หากผู้เล่นคนอื่นยังคงกลยุทธ์เดิมอยู่

\section{Quantum Mechanics}
\subsection{Superposition}
หลักการที่ระบบควอนตัมสามารถอยู่ในหลายสถานะได้พร้อมกัน จนกว่าจะมีการวัดหรือสังเกต

\subsection{Entanglement}
ปรากฏการณ์ที่อนุภาคควอนตัมถูกเชื่อมโยงกัน ทำให้การวัดอนุภาคหนึ่งส่งผลต่ออีกอนุภาคหนึ่งทันที

\subsection{Tunneling}
ปรากฏการณ์ที่อนุภาคสามารถผ่านอุปสรรคพลังงานได้ แม้พลังงานจะต่ำกว่าความสูงของกำแพง

\subsection{Quantum Computing}
เครื่องคอมพิวเตอร์ที่ใช้หลักการควอนตัมในการคำนวณ เช่น Gate-based Quantum Computing และ Quantum Annealing

\subsection{Quantum Annealing}
กระบวนการคำนวณเชิงควอนตัมเพื่อหาค่าที่เหมาะสมที่สุด โดยใช้ Superposition และ Quantum Tunneling

\subsection{QUBO Formulation}
การกำหนดปัญหาให้อยู่ในรูปแบบ Quadratic Unconstrained Binary Optimization ซึ่งเป็นปัญหา NP-hard

\subsection{Optimization Problems}
ปัญหาที่ต้องการหาคำตอบที่ดีที่สุดจากฟังก์ชันวัตถุประสงค์ โดยอาจมีหรือไม่มีข้อจำกัด

\section{\ifenglish%
\ifcpe CPE \else ISNE \fi knowledge used, applied, or integrated in this project
\else%
ความรู้ตามหลักสูตรซึ่งถูกนำมาใช้หรือบูรณาการในโครงงาน
\fi
}

\subsection{Algorithms for Computer Engineers} นํามาใช้ในการพัฒนาอัลกอริทึมในแบบจำลองควอนตัม
\subsection{Data Structures for Computer Engineers} นํามาใช้ในการออกแบบโครงสร้างแผนภาพต้นไม้ของปัญหา
\subsection{Discrete Math for Computer Engineers} นํามาใช้ในการพิสูจน์ทางตรรกศาสตร์ ตารางค่าความจริงในสมการ
\subsection{Advance Algorithm} นํามาใช้เป็นแนวทางการออกแบบ และวิเคราะห์อัลกอริทึมที่ให้ผลลัพธ์ที่เหมาะสมที่สุดสำหรับปัญหาทางไซเบอร์
\subsection{Quantum Computing} นำมาใช้เป็นกระบวนการหลักในการสร้างอัลกอริทึมในการคำนวณเชิงควอนตัม
\subsection{Penetration Testing} นำมาใช้ในการจำแนกประเภทของวิธีการโจมตีระบบความปลอดภัย
\subsection{Defensive Security} นำมาใช้ในการจำแนกประเภทของวิธีการป้องกันระบบความปลอดภัย

\section{\ifenglish%
Extracurricular knowledge used, applied, or integrated in this project
\else%
ความรู้นอกหลักสูตรซึ่งถูกนำมาใช้หรือบูรณาการในโครงงาน
\fi
}
\subsection{Game Theory (ทฤษฎีเกม)} ใช้เป็นทฤษฎีหลักในการประยุกต์กับปัญหาระบบความปลอดภัย
\subsection{Nash Equilibrium (ดุลยภาพแนช)} ใช้เพื่อวิเคราะห์ผลลัพธ์ที่ดีที่สุดของอัลกอริทึม
\subsection{Prisoner’s Dilemma (ปัญหานักโทษ)} ใช้ในการสร้างตัวต้นแบบก่อนพัฒนาอัลกอริทึมจริง
\subsection{Layered Security} ใช้อธิบายมาตรการป้องกันระบบความมั่นคงปลอดภัยไซเบอร์
